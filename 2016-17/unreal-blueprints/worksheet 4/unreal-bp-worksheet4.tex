\documentclass{../../../fal_assignment}
\graphicspath{ {../../../} }

\usepackage{enumitem}
\setlist{nosep} % Make enumerate / itemize lists more closely spaced
\usepackage[T1]{fontenc} % http://tex.stackexchange.com/a/17858
\usepackage{url}
\usepackage{todonotes}

\title{Worksheet Tasks}
\author{Brian McDonald} %based on Dr Ed Powley's COMP110 Worksheets
\module{Unreal Blueprints Worksheet}

\begin{document}

\maketitle

\begin{marginquote}
`We find you need to make a game wrong at least two or three times before you find the right path. ...
We took a lot of opportunity to design and explore, knowing that a lot of it would be thrown away.''
\par --- Ken Wong, lead designer \textbf{Monument Valley}
\end{marginquote}

\marginpicture{flavour_pic}{
    \emph{Unreal Engine} Blueprints
}

\section*{Introduction}
The goals of this series of workshops is to obtain a good working knowledge of Visual Scripting system in Unreal which is called Blueprints. This will enable you to not only create simple systems in a game (e.g. Triggers) but create full games.

Blueprints is the Visual Scripting language which is shipped inside Unreal Engine 4. This language is node based, where functions and variables can be represented as nodes on the graph and the relationship between these are defined by connections. These connections also allow you to easily see the flow of data through your code.

\section*{Objectives}
\begin{enumerate}[label=(\Alph*)]
	\item \textbf{Implement} An user interface using UMG
  \item \textbf{Explain} the basic makeup of a Particle system
  \item \textbf{Apply} your knowledge of UMG to create a main menu
\end{enumerate}

\section*{Worksheet Setup}

This is a continuation of worksheet 3, you should first finish off worksheet 4

\section*{Worksheet Tasks}

\begin{enumerate}
	\item As individuals carry out the following tasks
  	\begin{enumerate}[label=(\Alph*)]
      \item Follow this video to \textbf{create a UI} for your game \url{http://www.digitaltutors.com/tutorial/2085-Creating-UI-Elements-in-Unreal-Engine}
      \item \textbf{Create an explosion} when you destroy an enemy \url{https://www.youtube.com/watch?v=WDIhrn6XhHg}
      \item \textbf{Play a sound} when the enemy is destroyed \url{https://docs.unrealengine.com/latest/INT/BlueprintAPI/Audio/PlaySoundatLocation/index.html}
      \item \textbf{Add a door} to the level \url{https://wiki.unrealengine.com/Blueprint_Automated_Door_Tutorial}
      \item \textbf{Add an exit menu}, this should be triggered when the player's health reach 0. This menu will allow you to restart the game
    \end{enumerate}
\end{enumerate}

\section*{Additional Guidance}
When you are coding it might be difficult to figure out the state of the application, it is often a good idea to set a break point or use Print String function.


\section*{Additional Resources}

\begin{itemize}
  \item Introduction to Particle Systems \url{https://app.pluralsight.com/library/courses/introduction-particle-systems-unreal-engine-1880/table-of-contents}
  \item UI Animation \url{https://www.youtube.com/watch?v=F4QvCXO9XaQ}
  \item UMG Guide \url{https://docs.unrealengine.com/latest/INT/Engine/UMG/UserGuide/}
\end{itemize}

\end{document}
