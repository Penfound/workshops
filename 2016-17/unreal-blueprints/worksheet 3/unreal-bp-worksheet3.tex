\documentclass{../../../fal_assignment}
\graphicspath{ {../../../} }

\usepackage{enumitem}
\setlist{nosep} % Make enumerate / itemize lists more closely spaced
\usepackage[T1]{fontenc} % http://tex.stackexchange.com/a/17858
\usepackage{url}
\usepackage{todonotes}

\title{Worksheet Tasks}
\author{Brian McDonald} %based on Dr Ed Powley's COMP110 Worksheets
\module{Unreal Blueprints Worksheet}

\begin{document}

\maketitle

\begin{marginquote}
`We find you need to make a game wrong at least two or three times before you find the right path. ...
We took a lot of opportunity to design and explore, knowing that a lot of it would be thrown away.''
\par --- Ken Wong, lead designer \textbf{Monument Valley}
\end{marginquote}

\marginpicture{flavour_pic}{
    \emph{Unreal Engine} Blueprints
}

\section*{Introduction}
The goals of this series of workshops is to obtain a good working knowledge of Visual Scripting system in Unreal which is called Blueprints. This will enable you to not only create simple systems in a game (e.g. Triggers) but create full games.

Blueprints is the Visual Scripting language which is shipped inside Unreal Engine 4. This language is node based, where functions and variables can be represented as nodes on the graph and the relationship between these are defined by connections. These connections also allow you to easily see the flow of data through your code.

\section*{Objectives}
\begin{enumerate}[label=(\Alph*)]
	\item \textbf{Understand} the usage of actors and actor components
	\item \textbf{Implement} Unreal Actors using Blueprints
  \item \textbf{Apply} your knowledge of the collision system in unreal
  \item \textbf{Understand} how to debug Unreal Blueprints.
\end{enumerate}

\section*{Worksheet Setup}

This is a continuation of worksheet 2, you should add functionality into the project from worksheet 2.

\section*{Worksheet Tasks}

\begin{enumerate}
	\item As individuals add the following functionality into the project
  	\begin{enumerate}[label=(\Alph*)]
      \item Pickup item which adds health to the player.
      \item Pickup which adds more ammo to the player.
      \item A pickup which removes health from the player.
    \end{enumerate}
\item Split into Pairs (or Mobs) and refactor the code
		\begin{enumerate}[label=(\Alph*)]
		\item Use a base Interface for all pickups.
    \item Instead of health variable on FPS Player, add a health actor component.
    \item Instead of using an variable on FPS Player track ammo, add an ammo actor component.
    \item Add an enemy actor, attach the health component. Can you make the actor die when health hits zero?
    \item Add another enemy actor which shoots bullets at the player. When the player is hit take health away from the player.
		\end{enumerate}
\end{enumerate}

\section*{Additional Guidance}
When you are coding it might be difficult to figure out the state of the application, it is often a good idea to set a break point or use Print String function.


\section*{Additional Resources}

\begin{itemize}
  \item Introduction to Blueprints in Unreal \url{https://www.pluralsight.com/courses/introduction-blueprint-unreal-engine-1688}
  \item Creating Gameplay Systems in Unreal \url{https://www.pluralsight.com/courses/gameplay-systems-blueprint-features-unreal-engine-1861}
  \item Collision Detection in Unreal \url{https://docs.unrealengine.com/latest/INT/Engine/Physics/Collision/Overview/index.html}
  \item Casting \url{https://docs.unrealengine.com/latest/INT/Engine/Blueprints/UserGuide/CastNodes/}
  \item Interfaces \url{https://www.youtube.com/watch?v=fi1Ujpldy0g}
\end{itemize}

\end{document}
