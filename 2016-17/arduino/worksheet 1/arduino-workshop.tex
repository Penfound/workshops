\documentclass{../../../fal_assignment}
\graphicspath{ {../../../} }

\usepackage{enumitem}
\setlist{nosep} % Make enumerate / itemize lists more closely spaced
\usepackage[T1]{fontenc} % http://tex.stackexchange.com/a/17858
\usepackage{url}
\usepackage{todonotes}

\title{Worksheet Tasks}
\author{Alcwyn Parker} %based on Dr Ed Powley's COMP110 Worksheets
\module{Arduino}

\begin{document}

\maketitle

\begin{marginquote}
`We find you need to make a game wrong at least two or three times before you find the right path. ...
We took a lot of opportunity to design and explore, knowing that a lot of it would be thrown away.''
\par --- Ken Wong, lead designer \textbf{Monument Valley}
\end{marginquote}

\marginpicture{flavour_pic}{
    \emph{Arduino Uno} 
}

\section*{Introduction}
The goals of this series of workshops is to obtain a good working knowledge of the Arduino platform. This will enable you to break away from traditional forms of hardware used to interact with your games and instead create more playful and tangible experiences for your users to enjoy.

Arduino is not just a micro-controller. It is an entire open source community and ecosystem of hardware and software solutions that are designed from the ground up to be easy to learn. There are a variety of different Arduino boards and shields that will be useful no matter where your ideas take you, from wearables to motion tracking, smart rooms to retro interface. 

\section*{Objectives}
\begin{enumerate}[label=(\Alph*)]
	\item \textbf{Identify} the parts of the Arduino and their purpose 
	\item \textbf{Write} sketches and upload them to the Arduino board
	\item \textbf{Apply} the basics to create more complex relationships between sensors and actuators
	\item \textbf{Implement} a basic interface that communicates via the USB with Unreal Engine.
\end{enumerate}

\section*{Worksheet Setup}

In order to complete this worksheet you can either use the Arduino and hardware provided or simulate the tasks using the Arduino emulator at Circuits.io. The core tasks for this worksheet are:

\section*{Worksheet Tasks}
\begin{itemize}
	\item Familiarise yourself with the Arduino IDE \url{https://www.arduino.cc/en/Guide/Environment}
	\item Create a basic sketch that uses a button to control an LED \url{https://www.arduino.cc/en/Tutorial/Button}
	\item Exchange the switch for a potentiometer and use a conditional statement to decide whether the LED should be turned off or not. \url{https://programmingelectronics.com/tutorial-11-if-else-statement-comparison-operators-and-conditions/}
	\item Create a ladder of LEDs that act as a visualiser for the readings from a potentiometer. (extend the previous task)
	\item Write a sketch that sends the values from the potentiometer to the serial port to be received by the computer \url{https://www.arduino.cc/en/Tutorial/AnalogReadSerial}
	\item Use the values from the serial to change the intensity of a light within an Unreal or Unity game. 
		\begin{itemize}
			\item \url{http://playground.arduino.cc/Interfacing/CPPWindows} 
			\item Unreal \& Unity example provided by tutors \url{https://github.com/Falmouth-Games-Academy/workshops/tree/master/2016-17/arduino/worksheet%201/code}
			\item C++ Library: \url{https://github.com/wjwwood/serial}
			\item Worse case (plugin): \url{https://forums.unrealengine.com/showthread.php?68643-UE4Duino-Arduino-to-UE4-plugin-Release!}
		\end{itemize}
\end{itemize}


\section*{Additional Guidance}

The goal of this worksheet is to get you up to speed with the Arduino platform, most of the work carried out here will be will be fairly rudimentary and you will be expected to progress quickly and apply these tasks to your hardware interface design

If you are struggling with the Arduino tasks then you may find the resources listed below useful

\section*{Additional Resources}

\begin{itemize}
    \item Arduino \url{http://arduino.cc/}
    \item Adafruit tutorial \url{https://learn.adafruit.com/category/learn-arduino}
    \item Arduino video series \url{https://www.youtube.com/watch?v=fCxzA9_kg6s}
\end{itemize}

\end{document}
